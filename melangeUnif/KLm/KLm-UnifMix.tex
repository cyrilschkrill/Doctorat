\documentclass{article}
\usepackage[utf8]{inputenc} 

\usepackage{hyperref}

\usepackage[a4paper, total={6in, 8in}]{geometry}
\setlength{\parindent}{0em}

\usepackage{tikz}

\usepackage[french]{babel}  

\usepackage{amssymb}
\usepackage{amsmath}
\usepackage{amsthm}

\usepackage{float}

\usepackage{bigints}
\usepackage{relsize}

\usepackage{bbold}

\usepackage{cancel}

\usepackage{xcolor}

\makeatletter
\newcommand{\RemoveAlgoNumber}{\renewcommand{\fnum@algocf}{\AlCapSty{\AlCapFnt\algorithmcfname}}}
\newcommand{\RevertAlgoNumber}{\algocf@resetfnum}
\makeatother

\usepackage{minted}
\usepackage{newfloat}
\DeclareFloatingEnvironment{script} % new float <<<<


\newtheorem{theorem}{Théorème}
\newtheorem{contre-exemple}{Contre-exemple}
\newtheorem{corollary}{Corollaire}
\newtheorem{proposition}{Proposition}
\newtheorem{hyp}{Hypothèse}
\newtheorem{definition}{Définition}

\usepackage[ruled,vlined]{algorithm2e}
\usepackage{caption}

\newenvironment{fonction}[1][htb]
  {\renewcommand{\algorithmcfname}{Fonction}% Update algorithm name
   \begin{algorithm}[#1]%
  }{\end{algorithm}}

\definecolor{bg}{RGB}{22,43,58}


\title{KL$_m$ et Mélanges Uniformes}
\author{Cyril THOMMERET \\ LPSM - Sorbonne Université \\ SAFRAN Aircraft Engines}
\date{\today}



\begin{document}
    \maketitle
    \tableofcontents
    \vspace*{1.2cm}
    Nous avions défini, pour une fonction $\varphi$ génératrice d'une divergence, le critère suivant:
    $$ m_{\pi,\theta}:\, x\in\mathrm{supp}\,g\, \longmapsto \int(\varphi^\prime\circ\dfrac{g}{g_{\pi,\theta}})(t)\cdot{}g(t)\mathrm{d}t - (\varphi^\#\circ\dfrac{g}{g_{\pi,\theta}})(x)$$
    où $\varphi^\prime$ est la dérivée de $\varphi$ et où $\varphi^\#\equiv{}\mathrm{id}\circ\varphi^\prime - \varphi$. \\

    \section{Le contexte dans le cas KL$_m$}
    \subsection{Cas général}

    La fonction génératrice de la divergence de Kullback-Leiber \textit{modifiée} prend la forme suivante: 
    $$ \varphi_{KL_m}(x) := -\ln\,x+x-1 $$

    De sorte que sa dérivée soit égale à $\varphi_{KL_m}^\prime(x) = 1 - \frac{1}{x}$, et que donc,
    
    \begin{align*}
      \varphi^\#_{KL_m}(x)   :& = x\cdot\varphi_{KL_m}^\prime(x) + \varphi^\prime_{KL_m}(x) \\
                              & = x\cdot(1-\frac{1}{x}) + \ln{}x - x + 1 \\
                              & = \ln{}x
    \end{align*}



\end{document}
